\documentclass[12pt, a4paper]{report}
\usepackage[top=1cm, bottom=1cm, left=1cm, right=1cm]{geometry}

\usepackage[utf8]{inputenc}
\usepackage[russian]{babel}

\usepackage{array}
\newcolumntype{M}[1]{>{\centering\arraybackslash}m{#1}}

\begin{document}
	\pagenumbering{arabic}

	\begin{titlepage}
		\begin{center}
			\vspace*{1.5cm}

			\LARGE
			\textbf{ЭЛЕКТРОТЕХНИКА И ЭЛЕКТРОНИКА}

			\vspace{5cm}

			\Huge
			Рабочая тетрадь
		\end{center}
	\end{titlepage}
	
	\begin{center}
		Лабораторная работа №1
		
		\large ЛИНЕЙНАЯ ЭЛЕКТРИЧЕСКАЯ ЦЕПЬ ПОСТОЯННОГО ТОКА	

		\textbf{Цель работы}
	\end{center}
	Исследование цепи постоянного тока.
	\newline

	\begin{tabular}{||M{4cm} | M{1.5cm} | M{1.5cm} | M{1.5cm} | M{1.5cm} | M{1.5cm} | M{1.5cm}||}
		\hline 
		Параметры цепи & \( R_{load} = 0 \) & \( R_{load} = R_{line} = 100 Ом \) & \( R_{load} = R + 100 Ом \) & \( R_{load} = R + 300 Ом \) & \( R_{load} = R + 500 Ом \) & \( R_{load} = R = 700 Ом \) \\

		\hline
		Ток, I, А & 2 & 1 & 0.5405 & 0.3509 & 0.2597 & 0.2062 \\

		\hline 
		Мощность источника, \newline \( P_{source} = E*I, \) Вт & 400 & 200 & 108.1 & 70.18 & 51.94 & 41.24 \\

		\hline
		Мощность нагрузки, \newline \( P_{load} = I^2*R, \) Вт & 0 & 100 & 78.88 & 57.87 & 45.19 & 36.99 \\

		\hline
		К.П.Д. цепи, \newline \( \eta = \frac{P_{load}}{P_{source}}*100\% \) & 0 & 50 & 73 & 82 & 87 & 90 \\
		\hline
	\end{tabular}
\end{document}


